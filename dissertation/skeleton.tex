
\documentclass[bsc,frontabs,twoside,singlespacing,parskip,deptreport]{infthesis}

\begin{document}

\title{RegEx Search \& Replace Extension for Chrome and Firefox}

\author{Dalimil Hajek}

\course{Computer Science}
\project{BSc Hons Project Report}
\date{\today}

\abstract{
The aim of this project was to build a browser extension to allow users to search and replace text with regular expressions in editable text input fields of web pages. 

After evaluating existing extensions that were unsuccessfully attempting to implement this functionality, the new extension has been carefully designed, developed, and finally successfully released for Chrome and Firefox browsers.

In addition to the future-rich search and replace function, this plugin also adds the ability to save favorite patterns, store search history, or predefine text templates that can be inserted into the editable area of a page.

The software followed an iterative development process, where user feedback was collected via several means, including Google Analytics, which was used to track user interaction, and a support website used to collect user feedback comments.

After the initial release, about twenty updates have been subsequently released over the span of a few months. This iteration was further supported by automated tests of several kinds.

The extension has received excellent reviews and at the time of writing is installed on over 2000 devices.
}

\maketitle

\section*{Acknowledgements}
Acknowledgements go here. 

\tableofcontents

\chapter{Introduction}
% Introduction, Background, Design, Implementation, Evaluation and Conclusions

%in which the project topic is described and set in the context of published literature, and the main results are briefly summarized (about five pages). 

%This chapter should include a clear and concise summary of your contributions (examples: adapting a suite of existing code; interpreting a theoretical algorithm; coding; testing; conducting an experiment) preferably as a bulleted list.

%The Introduction chapter should always provide a 'roadmap' to the report. Of course it should provide an introduction to the problem being considered, but it should also give some details of what you did - do not leave this to the conclusion. You should give forward references into the rest of the report - e.g., "In Chapter 2 how algorithms and heuristics are used to deal with approximate counting are discussed", "The design of the system is presented in Chapter 4", "In Chapter 3 the reasons for choosing to focus on the bounded-degree case of this problem are explained". 

%Your report will be read by markers throughout the School of Informatics. Do not assume that your marker is an expert in your subfield. Give some basics as well as the details. Also, make sure you point out what was involved in solving certain problems; this can help a non-expert judge the work involved. For instance: "The interpolation algorithm was implemented in C directly, rather than using the routines in Matlab". "In order to develop a user interface appropriate for the educational system, 4 prototypes were tested on 50 students". 

% Use passive voice: A and B were used. instead of I used A and B

The document structure should include:
\begin{itemize}
\item
The title page  in the format used above.
\item
An optional acknowledgements page.
\item
The table of contents.
\item
The report text divided into chapters as appropriate.
\item
The bibliography.
\end{itemize}

Commands for generating the title page appear in the skeleton file and
are self explanatory.
The file also includes commands to choose your report type (project
report, thesis or dissertation) and degree.
These will be placed in the appropriate place in the title page. 

The default behaviour of the documentclass is to produce documents typeset in
12 point.  Regardless of the formatting system you use, 
it is recommended that you submit your thesis printed (or copied) 
double sided.

The report should be printed single-spaced.
It should be 30 to 60 pages long, and preferably no shorter than 20 pages.
Appendices are in addition to this and you should place detail
here which may be too much or not strictly necessary when reading the relevant section.

\section{Using Sections}

Divide your chapters into sub-parts as appropriate.

\section{Citations}

Note that citations 
(like \cite{P1} or \cite{P2})
can be generated using {\tt BibTeX} or by using the
{\tt thebibliography} environment. This makes sure that the
table of contents includes an entry for the bibliography.
Of course you may use any other method as well.

\section{Options}

There are various documentclass options, see the documentation.  Here we are
using an option ({\tt bsc} or {\tt minf}) to choose the degree type, plus:
\begin{itemize}
\item {\tt frontabs} (recommended) to put the abstract on the front page;
\item {\tt twoside} (recommended) to format for two-sided printing, with
  each chapter starting on a right-hand page;
\item {\tt singlespacing} (required) for single-spaced formating; and
\item {\tt parskip} (a matter of taste) which alters the paragraph formatting so that
paragraphs are separated by a vertical space, and there is no
indentation at the start of each paragraph.
\end{itemize}

\chapter{The Real Thing}

Of course
you may want to use several chapters and much more text than here.

%Discussion of the work undertaken, in which the various sub-problems, solutions and difficulties are examined. Discussion of any implementation should concentrate on overall principles and aspects of particular interest, rather than minute details.

%You should always make it clear what was completed, and what was left as future work. If in doubt spell it out: tell us which bits of code you could find already written, which bits you had to do from scratch, which bits were routine, which bits were challenging (and why). 

%If appropriate, a description of experiments undertaken, a presentation of the data gleaned from them, and an interpretation of that data. 

\chapter{Conclusion}
% in which the main achievements are reviewed, and unsolved problems and directions for further work are presented.

% use the following and \cite{} as above if you use BibTeX
% otherwise generate bibtem entries
\bibliographystyle{plain}
\bibliography{mybibfile}

\end{document}
