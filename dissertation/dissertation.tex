
\documentclass[bsc,frontabs,twoside,singlespacing,parskip,deptreport]{infthesis}
\usepackage{eurosym}
\usepackage{graphicx}
\usepackage{hyperref}
\hypersetup{unicode=true,
			colorlinks=true,
			linkcolor=black,
			urlcolor=cyan,
            pdfborder={0 0 0},
            breaklinks=true}
\urlstyle{same}
\providecommand{\tightlist}{%
  \setlength{\itemsep}{0pt}\setlength{\parskip}{0pt}}

\begin{document}

\title{RegEx Search \& Replace Extension for Chrome and Firefox}

\author{Dalimil Hajek}

\course{Computer Science}
\project{BSc Hons Project Report}
\date{\today}

\abstract{
The aim of this project was to build a browser extension to allow users to search and replace text with regular expressions in editable text input fields of web pages.

After evaluating existing extensions that were unsuccessfully attempting to implement this functionality, the new extension has been carefully designed, developed, and finally successfully released for Chrome and Firefox browsers.

In addition to the future-rich search and replace function, this plugin also adds the ability to save favorite patterns, store search history, or predefine text templates that can be inserted into the editable area of a page.

The software followed an iterative development process, where user feedback was collected via several means, including Google Analytics, which was used to track user interaction, and a support website used to collect user feedback comments.

After the initial release, about twenty updates have been subsequently released over the span of a few months. This iteration was further supported by automated tests of several kinds.

The extension has received excellent reviews and at the time of writing has over 3000 weekly users (users from both browsers combined).
}

\maketitle

\section*{Acknowledgements}
Thanks to
\begin{itemize}
\item
  Boris Grot - for supervising the whole project, and making important feature suggestions leading up to the first official releases of the extension on the Chrome and Firefox web stores
\item
  Michael O'Boyle - for making suggestions, especially regarding Google Analytics
\item
  Christoph Metze - for finding several important bugs that subsequently led to releases \texttt{1.3.2}, \texttt{1.3.3}, and \texttt{1.3.4}
\item
  Daniel Tomberlin - for pointing out a use case when trying to search across multiple single-line inputs, and for updating his web store rating and review after I implemented it in \texttt{1.3.6}
\item
  GitHub user \href{https://github.com/MarkRH}{MarkRH} - for finding a bug that was later fixed in \texttt{1.1.3}
\item
  StackOverflow user \href{https://stackoverflow.com/users/3959875/woxxom}{wOxxOm} - for suggesting \texttt{Document.execCommand} API that I used to fix issues with templates in \texttt{1.2.0}
\end{itemize}

And also thanks to all those people who submitted user feedback or
reviews.

\standarddeclaration

\tableofcontents
\listoffigures

\chapter{Introduction} % should be about 5 pages
Modern web browsers allow users to find text in a web page, but when it comes to editable text areas that are often used on blogging platforms, online forums, social media, and email web clients, as a form of user input, none of the existing browsers allow users to also replace the found occurrences.
 
The aim of this project was to build an extension that adds this browser functionality. The tool allows users to find and replace text in editable text input fields of web pages, and includes a large set of features and search options, including regular expression support and occurrence highlighting.

\section{Motivation}
Search and replace functionality can be extremely useful when composing long emails, writing posts on social media, online forums, or blogging platforms, as well as in any email web clients. 

The most frequent use cases are:
\begin{itemize}
\item Fixing a typographical error -- A word or a phrase may have been used several times with the wrong spelling. This can reoccur several times in a forum post or an email, and it would be convenient to replace everything at once.

This also includes fixing unreadable characters in blog entries such as {\it \^{A}\euro\r{?}}, due to a change in encoding or some unintended text handling.

\item Normalizing incorrectly formatted text -- Regular expressions can be used to detect formatting errors such as multiple spaces before a period, missing upper-case letter, various metric unit formatting errors, and similar. Search and replace extension supporting regular expressions can quickly find and fix these.

\item Renaming a phrase -- Often a word or a phrase that occurs several times throughout a text needs to be corrected or substituted (perhaps using a synonym or a wording that sounds better)
\end{itemize}

Without having a browser extension for search and replace, one could imagine a solution where all text is copied and pasted into an advanced text editor, fixed using the built-in search and replace function, and copied back into the web page input field.

In addition to being a lengthy and time consuming process, this method would in many cases lose all text formatting, because more advanced editable text elements on the web may contain images, emojis, and text containing many formatting tags, which would not be preserved during the copy-pasting.

Additional motivation behind the development of this project was to add search-and-replace related features that are missing even from the more advanced text editors. One of them is storing the search history, and also being able to save favorite search patterns, that can later be quickly accessed. Both of these would save time and increase user productivity.

\section{Existing Extensions}
Web browsers support standard search functionality for any text on a page but no browsers have the find \& replace functionality. Users have asked for this feature on Google Chrome forums\footnote{Google Chrome Forum link: \href{https://productforums.google.com/forum/\#!topic/chrome/Y4UORlpdYfo}{https://productforums.google.com/forum/\#!topic/chrome/Y4UORlpdYfo}}, but the decision of browser developers was to leave the implementation of this functionality to potential text-processing web applications, rather than implementing it as a part of the browser.

\begin{figure}[h]
\centering
\includegraphics[width=0.6\textwidth]{../docs/browser-find-toolbar/chrome-find.png}
\caption{Google Chrome browser find tool}
\end{figure}

\begin{figure}[h]
\centering
\includegraphics[width=0.6\textwidth]{../docs/browser-find-toolbar/firefox-find.png}
\caption{Mozilla Firefox browser find tool}
\end{figure}

There have been several attempts to implement this functionality via an extension. Most of them either don't work, are missing functionality (particularly support for regular expressions), are limited to certain websites, or are counter-intuitive and hard to use in general. 

\subsection{Chrome}

\begin{itemize}
\item
  {\bf Search and Replace}\footnote{\href{https://chrome.google.com/webstore/detail/search-and-replace/bldchfkhmnkoimaciljpilanilmbnofo}{https://chrome.google.com/webstore/detail/search-and-replace/bldchfkhmnkoimaciljpilanilmbnofo}}
  
  \begin{itemize}
  \tightlist
\item
  This is the dominant one with over 52,000 users (as of Dec 2017)
\item
  It has only $3.1/5$ stars and many reviews are complaining it doesn't work or that it destroys other content of the current page
\item
  It has usability issues (control UI is partially hidden) and it doesn't work in many places such as Blogger or Facebook
\item
  This is a very bad and simplistic solution that is mostly broken but benefits from a larger user base because it appeared in the web store in 2013
  \end{itemize}
  
\item
  {\bf Find Replace}\footnote{\href{https://chrome.google.com/webstore/detail/find-replace/cfjmfciolkikfodjfdmdpdmpfbjdofek}{https://chrome.google.com/webstore/detail/find-replace/cfjmfciolkikfodjfdmdpdmpfbjdofek}}
  \begin{itemize}
  \tightlist
\item
  This extension requires copy-pasting your desired text into the provided box (it is plain-text only, so formatting is lost).
\item
  It has almost 1000 users (as of Dec 2017), but it has been around since 2013
\item
  It is unable to perform the search and replace in browser. This is no better than copy pasting text into a text editor and performing the operation there
  \end{itemize}
\item
  \textbf{FindR}\footnote{\href{https://chrome.google.com/webstore/detail/findr/bidnaaogcagbdidehabnjfedabckhdgc}{https://chrome.google.com/webstore/detail/findr/bidnaaogcagbdidehabnjfedabckhdgc}}
  \begin{itemize}
  \tightlist
\item
  It has almost 2000 users and $3.9/5$ stars. It has been in the web store since April 2016.
\item
  This extension is used for replacing any HTML text in the page, rather than text in input fields -- for this purpose it seems to work, but when one tries to use it only for text input fields, things break and the extension stops working (highlighting and match indicator both disappear and replace button stops working)
\item
  It also requires permission to \textit{Read and change all your data on the websites that you visit}, which might be a privacy issue (the extension can read everything even when the user isn't using it)
  \end{itemize}
\item
  \textbf{Easy Replace}\footnote{\href{https://chrome.google.com/webstore/detail/easy-replace/ojoeejfegihohnkjlfoonbnailkohkce}{https://chrome.google.com/webstore/detail/easy-replace/ojoeejfegihohnkjlfoonbnailkohkce}}
  \begin{itemize}
  \tightlist
\item
  It has over 3000 users but only $2.6/5$ stars
\item
  Most of them report it doesn't work because it only focuses on plain-text text areas and completely ignores more advanced editable HTML elements that most sites use these days
  \end{itemize}
\end{itemize}

\subsection{Firefox}

\begin{itemize}
\item
  \textbf{Find and Replace for Firefox}\footnote{\href{https://addons.mozilla.org/en-US/firefox/addon/find-and-replace-for-firefox}{https://addons.mozilla.org/en-US/firefox/addon/find-and-replace-for-firefox}}
  \begin{itemize}
  \tightlist
\item
  Old add-on, not compatible with the latest Firefox 57.
\item
  It has almost 3000 users but only $3.5/5$ stars
\item
  It's not working for most users, has almost no options (no RegEx, no highlighting, etc.)
\item
  It was last updated in 2012 and is not maintained
  \end{itemize}
\item
  \textbf{FoxReplace}\footnote{\href{https://addons.mozilla.org/en-US/firefox/addon/foxreplace/}{https://addons.mozilla.org/en-US/firefox/addon/foxreplace/}}
  \begin{itemize}
  \tightlist
\item
  This extension provides different functionality -- it asks the user to predefine a list of substitutions and then it automatically applies them globally across the text in newly loaded web pages
\item
  FoxReplace has over 7000 users (Dec 2017) but targets a different audience
  \end{itemize}
\end{itemize}

\subsection{Other Browsers}
Supporting more browsers add extra work, so it only makes sense if these browsers represent a significant portion of the market. The usage share of browsers is measured by various methods and is often quite inaccurate.\footnote{\href{https://en.wikipedia.org/wiki/Usage\_share\_of\_web\_browsers}{https://en.wikipedia.org/wiki/Usage\_share\_of\_web\_browsers}} Approximate values for usage share for desktop browsers are:\footnote{\href{http://gs.statcounter.com/browser-market-share/desktop/worldwide/\#monthly-201712-201712-bar}{http://gs.statcounter.com/browser-market-share/desktop/worldwide/\#monthly-201712-201712-bar}} Chrome 65\%, Firefox 12\%, IE 8\%, Safari 6\%, Edge 4\%.

This project focused on Chrome and Firefox, which together represent a large portion of the overall market, and which both mostly follow the same Extension API (there are only a few differences\footnote{\href{https://developer.mozilla.org/en-US/Add-ons/WebExtensions/Chrome\_incompatibilities}{https://developer.mozilla.org/en-US/Add-ons/WebExtensions/Chrome\_incompatibilities}} so we didn't need to have separate code-bases).

Safari, although widely used by Mac users, has its own extension API and is in general much more involved as it requires dealing with Apple's developer libraries and licenses, so we decided not to develop an extension for this browser.


%This chapter should include a clear and concise summary of your contributions (examples: adapting a suite of existing code; interpreting a theoretical algorithm; coding; testing; conducting an experiment) preferably as a bulleted list.

%The Introduction chapter should always provide a 'roadmap' to the report. Of course it should provide an introduction to the problem being considered, but it should also give some details of what you did - do not leave this to the conclusion. You should give forward references into the rest of the report - e.g., "In Chapter 2 how algorithms and heuristics are used to deal with approximate counting are discussed", "The design of the system is presented in Chapter 4", "In Chapter 3 the reasons for choosing to focus on the bounded-degree case of this problem are explained". 

%Your report will be read by markers throughout the School of Informatics. Do not assume that your marker is an expert in your subfield. Give some basics as well as the details. Also, make sure you point out what was involved in solving certain problems; this can help a non-expert judge the work involved. For instance: "The interpolation algorithm was implemented in C directly, rather than using the routines in Matlab". "In order to develop a user interface appropriate for the educational system, 4 prototypes were tested on 50 students". 

% Use passive voice: A and B were used. instead of I used A and B


\chapter{Background}
%  A brief introduction to browser extensions would be a good addition to an early part of the thesis. 

\chapter{Design}

\section{Naming and Search Engine Optimization}
Based on the competition research mentioned in the introduction chapter above, the following extension names already exist: \textit{Easy Replace}, \textit{Search and Replace}, \textit{FindR}, \textit{Find Replace}, \textit{FoxReplace}. Using any of these existing names would be bad for the SEO and discoverability, and would most likely lead to confusions.

At the same time, we want to clearly indicate that the extension is used for input fields and editable content rather than HTML source code.

People are likely to search for browser extensions by typing in the functionality that they need. In our case, that might look something like "find and replace in text input fields extension". Stating the extensions purpose in its title and description should help us do better in search results. Therefore, we avoided any newly-invented words and named it \textit{Find \& Replace for Text Editing}. We are not trying to trademark a new brand name, we are simply trying to match what people might search for, so that's why this name was chosen.

The naming idea was largely successful. In January 2018 (a few months after the first release), the extension ranked 2nd in the Google search results for queries \textit{search and replace in chrome}, and \textit{search and replace browser}, 1st for the query \textit{search and replace in firefox}, 7th for the query \textit{find and replace text}, and 8th for the query \textit{search and replace text}.\footnote{Tested using incognito modes in both Firefox and Chrome, and searching via www.google.co.uk}

\section{User Interface Inspiration}
To some extent, we wanted to follow the current standard of find \& replace toolbars. Many of these can be seen in more advanced text editors. The user interfaces of some of the popular ones were examined and used as an inspiration.

\begin{figure}[hp]
\centering
\includegraphics[width=0.9\textwidth]{../docs/editor-find-and-replace/android-studio-find-and-replace.png}
\caption{Android Studio find and replace tool}
\end{figure}

\begin{figure}[hp]
\centering
\includegraphics[width=0.9\textwidth]{../docs/editor-find-and-replace/gdocs-find-and-replace.png}
\caption{Google Docs find and replace tool}
\end{figure}

\begin{figure}[hp]
\centering
\includegraphics[width=0.9\textwidth]{../docs/editor-find-and-replace/vscode-find-and-replace.png}
\caption{Visual Studio Code find and replace tool}
\end{figure}

\begin{figure}[hp]
\centering
\includegraphics[width=0.9\textwidth]{../docs/editor-find-and-replace/sublime-find-and-replace.png}
\caption{Sublime Text 3 find and replace tool}
\end{figure}

We wanted our extension to follow some of the design patterns that the existing text editors use, so that users are presented with a user interface that they can easily understand and start using quickly.

At the same time, it should not be assumed that average users are familiar with regular expression or more advanced search functions. Therefore, the UI design of some of these editor widgets should only be used as an inspiration - average users are not developers and the number of options in this extension must not feel overwhelming.

\section{Components}
To support all the functionality that we specified, the extension widget should have the following UI elements:
\begin{itemize}
\tightlist
\item
  \textit{Find} input field
\item
  \textit{Replace} input field
\item
  \textbf{Action buttons}

  \begin{itemize}
  \tightlist
  \item
    Replace
  \item
    Replace All
  \item
    Find next
  \item
    Find previous
  \item
    Save to favorites
  \item
  	Expand/Collapse advanced search options
  \end{itemize}
\item
  \textbf{Options}

  \begin{itemize}
  \tightlist
  \item
    Match Case (Aa)
  \item
    Whole Word (Ab\textbar{})
  \item
    Use Regex (.*)
  \item
    Limit to Text Selection
  \end{itemize}
\item
  \textit{X of Y} or \textit{No Results} indicator
\item
  Regex groups indicator (for regex search only)
\item
  \textbf{Panel tabs}

  \begin{itemize}
  \item
    Favorites
  \item
    History
  \item
    Templates
  \item
    Help/Info/Feedback
  \end{itemize}
\end{itemize}

\chapter{Implementation}
% overall principles and aspects of particular interest, rather than minute details.

%You should always make it clear what was completed, and what was left as future work. If in doubt spell it out: tell us which bits of code you could find already written, which bits you had to do from scratch, which bits were routine, which bits were challenging (and why). 

\chapter{Evaluation}
% user iteration

\chapter{Conclusions}
% in which the main achievements are reviewed, and unsolved problems and directions for further work are presented.

% Bibliography section --- use \cite{P1} with BibTeX
\bibliographystyle{plain}
\bibliography{mybibfile}

\end{document}
